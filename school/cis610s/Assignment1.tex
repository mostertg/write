\documentclass[]{article}
\usepackage{lmodern}
\usepackage{amssymb,amsmath}
\usepackage{ifxetex,ifluatex}
\usepackage{fixltx2e} % provides \textsubscript
\ifnum 0\ifxetex 1\fi\ifluatex 1\fi=0 % if pdftex
  \usepackage[T1]{fontenc}
  \usepackage[utf8]{inputenc}
\else % if luatex or xelatex
  \ifxetex
    \usepackage{mathspec}
  \else
    \usepackage{fontspec}
  \fi
  \defaultfontfeatures{Ligatures=TeX,Scale=MatchLowercase}
\fi
% use upquote if available, for straight quotes in verbatim environments
\IfFileExists{upquote.sty}{\usepackage{upquote}}{}
% use microtype if available
\IfFileExists{microtype.sty}{%
\usepackage{microtype}
\UseMicrotypeSet[protrusion]{basicmath} % disable protrusion for tt fonts
}{}
\usepackage[unicode=true]{hyperref}
\hypersetup{
            pdfborder={0 0 0},
            breaklinks=true}
\urlstyle{same}  % don't use monospace font for urls
\IfFileExists{parskip.sty}{%
\usepackage{parskip}
}{% else
\setlength{\parindent}{0pt}
\setlength{\parskip}{6pt plus 2pt minus 1pt}
}
\setlength{\emergencystretch}{3em}  % prevent overfull lines
\providecommand{\tightlist}{%
  \setlength{\itemsep}{0pt}\setlength{\parskip}{0pt}}
\setcounter{secnumdepth}{0}
% Redefines (sub)paragraphs to behave more like sections
\ifx\paragraph\undefined\else
\let\oldparagraph\paragraph
\renewcommand{\paragraph}[1]{\oldparagraph{#1}\mbox{}}
\fi
\ifx\subparagraph\undefined\else
\let\oldsubparagraph\subparagraph
\renewcommand{\subparagraph}[1]{\oldsubparagraph{#1}\mbox{}}
\fi

% set default figure placement to htbp
\makeatletter
\def\fps@figure{htbp}
\makeatother


\date{}

\begin{document}

\subsection{Introduction}\label{introduction}

The Cooperative Governance model is a rather new approach to corporate
governance and governance in general. As a framework it is a subset in
the larger corporate governance concept, but sets itself apart from the
base by diverging from long established ideals of governance, and by
embracing a more humanistic and democratic aspect of organizational
governance.

This essay will critique the assertion that the cooperative model
embraces a democratic approach to corporate governance that is not
unlike the constitution. And like the constitution, this model of
organizational management is not without its wrinkles and creases making
it difficult and/or controversial to implement in businesses.

\subsection{Features of Cooperative Governance
model}\label{features-of-cooperative-governance-model}

As legal entities within society, corporations are expected to operate
within certain parameters of society such that the business performs as
expected and turns a profit, but does so within acceptable ethical
standards. The purpose of corporate governance is to ensure that
organizations and institutions adhere to these set rules, conforming to
the principles of corporate governance by maintaining ``integrity and
ethical behaviour, accountability and transparency {[}as well as{]}
defining the role and responsibility of the board of management
\citep{cis17}. Traditionally, the governance of institutions is
conceptualised as the division between (1) policy and strategy and (2)
execution of set policies and strategies. This is exemplified in the
corporate governance structure of managers, who are external
stakeholders tasked by shareholders to run and manage the operations of
the organization such that it conforms to the ideals, policy and
strategies set by owners and/or their representatives, the board.

Cooperative governance structures do not have this division though, as
the cooperative framework applied the approach of owner-managed
organizations. Here the model applies the principle that shareholders
are also managers that own and run the institution. Structure is not
based on capital, but rather on an individual basis, thus providing more
egalitarian principles. Furthermore, the voting and decision structure
of the cooperative is much more simplified, with each member having one
vote. This democratises the decision making process in the organisation,
as the model is based on membership of the individual, and not
membership of capital as in the traditional corporate model.

\subsection{Advantages of Cooperative
model}\label{advantages-of-cooperative-model}

The Cooperative Governance model is a much newer conceptualisation of
governance in the context of institutions such as corporations, but it
offers the possibility of several advantages.

Firstly, the ownership and control structure is much simpler because it
is operated and owned by members of the cooperative. Membership is based
on production purchase, and not capital ownership. Decision making is
also more democratic, since each member has one vote, but decision
making includes every member \citep{chron}. This allows even the
smallest producer member to have a say in his/her future, and provides a
much more egalitarian approach to conducting business. Furthermore,
owner-members can be assured that policy directives are carried out in
alignment with member vision \citep{bancobr}.

Cooperative-run organizations tend to be smaller, and thus more agile in
terms of business functionality and can respond quicker to changes in
the market. This however does not diminish the purchasing power of the
cooperative -- members can leverage this effective buying power and use
it obtain raw produce at lower prices. Agra \citep{agra}, started with a
cooperative model of governance, and until recently (when they became a
private corporation, mainly due to size considerations) operated
effectively in the market.

\subsection{Problematic areas of the Cooperative
model}\label{problematic-areas-of-the-cooperative-model}

Traditionally, the cooperative model has provided a very open structure,
allowing the organisation to be very transparent to shareholders. This
however can be a deal-breaker for organizations that operate in
industries that are highly regulated due to privacy and/or ethical
considerations. Industries such as telecommunications, banking and
medical service providers cannot by law, comply with cooperative
governance models due to the lack of secrecy involved in this framework
\citep{yal}.

Cooperative institutions can be (correctly or incorrectly) perceived as
having little or no business acumen \citep{yal}, therefore resorting to
the use of of the cooperative model as governance framework. This can
have an impact on the ability of the organisation to procure funding
through the use of loans.

Another disadvantage of this model is the.

\subsection{Conclusion}\label{conclusion}

The cooperative governance model for organisations is a very new
concept, and certainly not much explored fully in the Namibian context.
This is evident with Agra choosing to pursue a more traditional approach
when it incorporated into a private institution \citep{agra}.

This however does not mean that it is not viable, as the model can be
seen to be very democratic in nature, and hence very disruptive and
effective. While no such company exists, one could hypothetically
construct a Namibian entity that is managed according to the cooperative
governance structure. Ideally, such an organisation would be owned by
each and every Namibian, thereby empowering the ordinary citizen, and
indirectly addressing the challenge of poverty eradication.

Such a hypothetical institution need not be only a silly construct, as
although the ownership would fall to citizens, management and strategy
decisions would be run via a nominated Board of Administration, with
yearly shareholder meetings similar to that of larger corporate
entities. Note however, out hypothetical institution need not be thought
of as a state-run entity - contrary, state-run entities behave much
similarly to traditional corporate institutions.

\end{document}
